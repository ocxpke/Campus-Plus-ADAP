\documentclass[11pt, a4paper]{article} % Declaramos que vamos a trabajar con un documento de tipo articulo
\usepackage[utf8]{inputenc} % Para que latex reconozca caracteres especiales
\usepackage[T1]{fontenc} % Mejora la forma en la que aparecen caracteres acentuados
\usepackage[spanish]{babel} % Carga las característica de la lengua española
\usepackage{graphicx} % Para incluir imágenes
\usepackage[colorlinks=true, linkcolor=black]{hyperref}
\usepackage[hmargin=3cm]{geometry} % Para el margen horizontal (espacio en blanco que hay entre bordes y texto)

\title{Práctica 1 - Extracción de Requisitos y Casos de Uso: Campus Plus}

\author{Jose Ramírez Girón}

\date{\today}

\begin{document}

\maketitle

\newpage

\tableofcontents

\newpage

\section{Introducción}
	En este caso nos encontramos con una escuela de clases de apoyo para la universidad llamada Campus Plus.
	En esta nos dicen el director, Antonio Ramírez, que tienen muchos problemas con diferentes gestiones como
	pueden ser:
	\begin{enumerate}
		\item La gestión del tablón de anuncios ya sea para la publicación de contenido, como la visualización de este.
		\item La gestión del horarios pues hay conflicto de este con versiones previas.
		\item Un servicio de gestión de archivos pues suele haber conflictos con las versiones y con como los usuarios
		encuentran estos archivos.
		\item Gestión de perfiles de usuario, ya sea modificación, creación, o el borrado de estos.
		\item Gestión de pagos, comprobantes, facturas, y pago fraccionado.
		\item Creación de copias de seguridad.
	\end{enumerate}

	Todos estos servicios han de incluirse y unificarse de forma que todos los diferentes tipos de usuario puedan
	llegar a usar la aplicación sin ningún tipo de problema. Es decir que todos los tipos de usuario han de convivir
	dentro del sistema sin problemas. \newline
	Ademas hay que hacer que el sistema se comunique con una pasarela de pago, para la gestión de estos, ademas de poder
	permitir un pago fraccionado. De esta forma los estudiantes podrán pagar de forma mas organizada y controlada.\newline
	También se ha de añadir que se busca mas seguridad y control sobre los archivos "importantes", es decir: Copias de
	seguridad sobre perfiles de alumnos, datos criticos (y su correspondiente acceso). Se buscan servicios como las
	pasarelas de pago y procesos de autentificación.

\section{Actores}
	Hemos considerado que para representar los diferentes tipos de usuario que tendrá nuestro sistema trabajaremos con
	los siguientes actores:
	\begin{itemize}
		\item \textit{Estudiantes:} \newline Consideramos que los estudiantes han de ser
		unos actores pues han de interactuar con el sistema de diferentes formas como pueden ser
		acceder al tablón de anuncios, comprobar el horario de las clases, gestionar sus pagos (al pagar reciben su factura),
		poder escribir y leer mensajes en el servicio de mensajería, y por ultimo el como pueden acceder a los materiales
		de cada asignatura.
		\item \textit{Profesor:} \newline Los profesores por hechos similares al de los estudiantes, revisar el horario,
		mensajería (recepción y envió de mensajes), administración del materia de la asignatura impartida por cada uno,
		publicar en el tablón de anuncios, etc... . Añadir también una forma de comprobar la asistencia de los alumnos.
		\item \textit{Servicio técnico:} \newline El servicio técnico ha de poder generar copias de seguridad de forma
		inmediata, gestionar los diferentes permisos y accesos, ¿gestión de flujo de usuarios?, gestión de usuarios y
		perfiles.
		\item \textit{Secretaria Administrativa:} \newline La secretaria interactuaría con el sistema en puntos como la
		gestión de usuarios, creación de perfiles, modificación de perfiles, gestión de pagos, gestión del tablón de
		anuncios y horarios.
		\item \textit{Director Académico:} \newline Y por ultimo consideramos que el director debe tener la posibilidad
		de acceder a datos de usuarios, modificación del tablo de anuncio, publicación de horarios,
	\end{itemize}

\section{Clasificación de requisitos}
	Vamos a enumerar y describir los requisitos funcionales y no funcionales que tendra el sistema que diseñaremos.
	Con los \textbf{requisitos funcionales} podremos describir la el comportamiento que tendra nuestro sistema ante
	ciertos eventos para que se satisfagan las necesidades del usuario. Mientras que con los
	\textbf{requisitos no funcionales} describir características "pasivas" de nuestro sistema, como pueden ser: el
	tiempo de respuesta, compatibilidad movil, web responsive, etc...
\subsection{Requisitos funcionales}
	\begin{enumerate}

		\item \textbf{RF-01: Inicio de sesion usuario.}
			\begin{itemize}
				\item \textit{Como: } Estudiante.
				\item \textit{Quiero: } Iniciar sesion con mis credenciales personales.
				\item \textit{Para: } Poder acceder a mi temario, asignaturas, y servicio de mensajería personal.
			\end{itemize}

		\item \textbf{RF-02: Ver tablon de anuncio.}
			\begin{itemize}
				\item \textit{Como: }Estudiante, profesores, servicio IT, secretaria y director academico.
				\item \textit{Quiero: } Poder ver que mensajes hay publicados en el tablon de anuncios.
				\item \textit{Para: } Poder estar al dia de las ultimas notificaciones, y eventos.
			\end{itemize}

		\item \textbf{RF-03: Ver horario de clases.}
			\begin{itemize}
				\item \textit{Como: }Estudiante, profesores, secretaria y director academico.
				\item \textit{Quiero: } Poder ver el horario unificado de todas las clases.
				\item \textit{Para: } Para saber donde tengo que ir y poder estar bine organizado.
			\end{itemize}

		\item \textbf{RF-04: Modificar horario.}
			\begin{itemize}
				\item \textit{Como: } Secretaria y director academico
				\item \textit{Quiero: } Poder actualizar el horario academico ya publicado.
				\item \textit{Para: } Que no haya confusion entre alumnos y profesores sobre a que horas y aulas son las clases.
			\end{itemize}

		\item \textbf{RF-05: Publicar anuncio.}
			\begin{itemize}
				\item \textit{Como: } Profesores, servicio IT, secretaria y director academico.
				\item \textit{Quiero: } Poder publicar un nuevo anuncio relacionado con algun nuevo evento, actualizacion de clases, etc...
				\item \textit{Para: } Que los alumnos, profesores o publico objetivo del anuncio pueda estar al tanto.
			\end{itemize}

		\item \textbf{RF-06: Inscribir alumno.}
			\begin{itemize}
				\item \textit{Como: }Secretaria.
				\item \textit{Quiero: } Poder registrar a un nuevo en el alumno.
				\item \textit{Para: } Para que el nuevo alumno pueda tener sus credenciales de acceso, y que nosotros podamos tener la informacion necesaria del alumno.
			\end{itemize}

		\item \textbf{RF-07: Ver los mensajes recibidos.}
			\begin{itemize}
				\item \textit{Como: }Estudiante, profesores, servicio IT, secretaria y director academico.
				\item \textit{Quiero: } Poder ver que mensajes he recibido  de forma personal, o grupal.
				\item \textit{Para: } Poder ver los mensajes donde soy de forma directa un receptor de este.
			\end{itemize}

		\item \textbf{RF-08: Enviar un nuevo mensaje.}
			\begin{itemize}
				\item \textit{Como: }Estudiante, profesores, servicio IT, secretaria y director academico.
				\item \textit{Quiero: } Poder enviar un nuevo mensaje a algun destinatario paraticular, o un grupo de destinatarios.
				\item \textit{Para: } Poder mantener una conversacion en particular o actualizar cierto tema a un particular/grupo.
			\end{itemize}

		\item \textbf{RF-09: Ver los mensajes recibidos.}
			\begin{itemize}
				\item \textit{Como: }Estudiante, profesores, servicio IT, secretaria y director academico.
				\item \textit{Quiero: } Poder ver que mensajes he recibido  de forma personal, o grupal.
				\item \textit{Para: } Poder ver los mensajes donde soy de forma directa un receptor de este.
			\end{itemize}

		\item \textbf{RF-10: Modificar la informacion de un alumno.}
			\begin{itemize}
				\item \textit{Como: }Secretaria.
				\item \textit{Quiero: } Poder modificar la informacion de un perfil de tipo estudiante.
				\item \textit{Para: } Para que el alumno tenga su informacion personal actualizada.
			\end{itemize}

		\item \textbf{RF-11: Eliminar un alumno.}
			\begin{itemize}
				\item \textit{Como: }Secretaria.
				\item \textit{Quiero: } Eliminar el perfil de un alumno de la base de datos.
				\item \textit{Para: } Eliminar un informacion no necesaria de la base de datos.
			\end{itemize}


	\end{enumerate}
\subsection{Requisitos no funcionales}
\section{Diagramas de casos de uso}
\section{Casos de uso criticos}

\section{Descripciones de casos de uso}

\end{document}

